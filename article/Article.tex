%
% Complete documentation on the extended LaTeX markup used for Insight
% documentation is available in ``Documenting Insight'', which is part
% of the standard documentation for Insight.  It may be found online
% at:
%
%     http://www.itk.org/

\documentclass{InsightArticle}


%%%%%%%%%%%%%%%%%%%%%%%%%%%%%%%%%%%%%%%%%%%%%%%%%%%%%%%%%%%%%%%%%%
%
%  hyperref should be the last package to be loaded.
%
%%%%%%%%%%%%%%%%%%%%%%%%%%%%%%%%%%%%%%%%%%%%%%%%%%%%%%%%%%%%%%%%%%
\usepackage[dvips,
bookmarks,
bookmarksopen,
backref,
colorlinks,linkcolor={blue},citecolor={blue},urlcolor={blue},
]{hyperref}
% to be able to use options in graphics
\usepackage{graphicx}
% for pseudo code
\usepackage{listings}
% subfigures
\usepackage{subfigure}


%  This is a template for Papers to the Insight Journal. 
%  It is comparable to a technical report format.

% The title should be descriptive enough for people to be able to find
% the relevant document. 
\title{Exponentiation filters}

% 
% NOTE: This is the last number of the "handle" URL that 
% The Insight Journal assigns to your paper as part of the
% submission process. Please replace the number "1338" with
% the actual handle number that you get assigned.
%
\newcommand{\IJhandlerIDnumber}{3227}

% Increment the release number whenever significant changes are made.
% The author and/or editor can define 'significant' however they like.
% \release{0.00}

% At minimum, give your name and an email address.  You can include a
% snail-mail address if you like.
\author{Ga\"etan Lehmann}
\authoraddress{INRA, UMR 1198; ENVA; CNRS, FRE 2857, Biologie du D\'eveloppement et Reproduction, Jouy en Josas, F-78350, France.}

\begin{document}


%
% Add hyperlink to the web location and license of the paper.
% The argument of this command is the handler identifier given
% by the Insight Journal to this paper.
% 
\IJhandlefooter{\IJhandlerIDnumber}

\maketitle

\ifhtml
\chapter*{Front Matter\label{front}}
\fi


\begin{abstract}
\noindent
Two simple filters are contributed to compute the exponential of an image: the first one to raise the image to the power of a constant, the
other one to raise the image to the power of the values provided in another image.
\end{abstract}

\IJhandlenote{\IJhandlerIDnumber}

\tableofcontents


\section{ConstantPowImageFilter}

\verb$ConstantPowImageFilter$ raises the input image to the power of a constant provided by the user.
Here is an example of usage:
\small \begin{verbatim}
#include "itkImageFileReader.h"
#include "itkImageFileWriter.h"
#include "itkSimpleFilterWatcher.h"

#include "itkConstantPowImageFilter.h"


int main(int argc, char * argv[])
{

  if( argc != 4 )
    {
    std::cerr << "usage: " << argv[0] << " intput output constant" << std::endl;
    std::cerr << " input: the input image" << std::endl;
    std::cerr << " output: the output image" << std::endl;
    // std::cerr << "  : " << std::endl;
    exit(1);
    }

  const int dim = 2;
  
  typedef unsigned char PType;
  typedef itk::Image< PType, dim > IType;
  typedef itk::Image< unsigned short, dim > OType;

  typedef itk::ImageFileReader< IType > ReaderType;
  ReaderType::Pointer reader = ReaderType::New();
  reader->SetFileName( argv[1] );

  typedef itk::ConstantPowImageFilter< IType, double, OType > FilterType;
  FilterType::Pointer filter = FilterType::New();
  filter->SetInput( reader->GetOutput() );
  filter->SetConstant( atof(argv[3]) );
  itk::SimpleFilterWatcher watcher(filter, "filter");

  typedef itk::ImageFileWriter< OType > WriterType;
  WriterType::Pointer writer = WriterType::New();
  writer->SetInput( filter->GetOutput() );
  writer->SetFileName( argv[2] );
  writer->Update();

  return 0;
}
\end{verbatim} \normalsize

\section{PowImageFilter}
\verb$PowImageFilter$ raises the first input image to the power of the values of the second image. Here is an example of usage:
\small \begin{verbatim}
#include "itkImageFileReader.h"
#include "itkImageFileWriter.h"
#include "itkSimpleFilterWatcher.h"

#include "itkPowImageFilter.h"


int main(int argc, char * argv[])
{

  if( argc != 4 )
    {
    std::cerr << "usage: " << argv[0] << " intput pow_input output" << std::endl;
    std::cerr << " input: the input image" << std::endl;
    std::cerr << " output: the output image" << std::endl;
    // std::cerr << "  : " << std::endl;
    exit(1);
    }

  const int dim = 2;
  
  typedef unsigned char PType;
  typedef itk::Image< PType, dim > IType;
  typedef itk::Image< unsigned short, dim > OType;

  typedef itk::ImageFileReader< IType > ReaderType;
  ReaderType::Pointer reader = ReaderType::New();
  reader->SetFileName( argv[1] );

  ReaderType::Pointer reader2 = ReaderType::New();
  reader2->SetFileName( argv[2] );

  typedef itk::PowImageFilter< IType, IType, OType > FilterType;
  FilterType::Pointer filter = FilterType::New();
  filter->SetInput( reader->GetOutput() );
  filter->SetInput( 1, reader2->GetOutput() );
  itk::SimpleFilterWatcher watcher(filter, "filter");

  typedef itk::ImageFileWriter< OType > WriterType;
  WriterType::Pointer writer = WriterType::New();
  writer->SetInput( filter->GetOutput() );
  writer->SetFileName( argv[3] );
  writer->Update();

  return 0;
}
\end{verbatim} \normalsize

\section{Wrapping}

The wrappers for WrapITK for the two new filters are provided in the Wrapping directory.

\appendix



\bibliographystyle{plain}
\bibliography{InsightJournal}
\nocite{ITKSoftwareGuide}

\end{document}

